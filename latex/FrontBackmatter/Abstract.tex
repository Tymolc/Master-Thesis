%*******************************************************
% Abstract
%*******************************************************
%\renewcommand{\abstractname}{Abstract}
\pdfbookmark[1]{Abstract}{Abstract}
\begingroup
\let\cleardoublepage\relax
\let\cleardoublepage\relax

\chapter*{Abstract}
With a growing 3D printing community and shrinking prices of 3D printers, personal fabrication is already a topical subject in the consumer market. However, current fabrication methods are limited to small-sized objects. Large-scale 3D printing is still a privilege to industry. In this thesis, we explore how to create large-scale kinetic objects using desktop-sized 3D printers and plastic bottles. We are building on node-link structures, also known as trusses, that can deform using linear actuators. Our end-to-end system \textit{TrussFormer} allows users to design and build such structures. TrussFormer incorporates an editor, which helps users create truss objects, even without engineering knowledge, a physics engine, which can simulate and visualize the substantial forces large objects produce and an export functionality, which lets users recreate their designed objects in the real world.\\
\\
Our system supports users by ensuring that designed objects are mechanically sound and will not break during movement. It generates all required parts for fabricating the object. We developed a custom hinge system that allows the introduction of motion. The parts can be created using desktop-sized 3D printers.\\
\\
We demonstrate TrussFormer with several example objects, including a walking 6-legged robot and a 4m tall animatronic T-Rex.

\clearpage

\begin{otherlanguage}{ngerman}
\pdfbookmark[1]{Zusammenfassung}{Zusammenfassung}
\chapter*{Zusammenfassung}
Durch eine wachsende 3D-Druck Gemeinschaft und sinkenden Preisen von 3D Druckern ist ``Personal Fabrication'' bereits ein aktuelles Thema im Verbraucher Markt. Die Fertigungsmethoden sind allerdings auf relativ kleine Objekte beschränkt. 3D Druck großer Objekte ist noch immer der Industrie vorbehalten. In dieser Masterarbeit erforschen wir, wie große kinetische Objekte mithilfe von handelsüblichen 3D Druckern und Plastikflaschen erstellt werden können. Wir setzen auf Knoten-Verbindungs-Strukturen, auch als Fachwerk bekannt, welche sich mithilfe von Lineargetrieben verformen können. Unser Ende-zu-Ende-System \textit{TrussFormer} ermöglicht Nutzer solche Strukturen zu gestalten und zu bauen. TrussFormer umfasst einen Editor, welcher Netzer dabei hilft solche Fachwerk-Strukturen zu erzeugen, eine Physik-Engine, welche die hohen Kräfte, die große Objekte erzeugen, simuliert und eine Export-Funktionalitäte, mithilfe welcher Nutzer ihre gestalteten Objekte nachbauen können.\\
\\
Unser System unterstützt Nutzer, indem es sicherstellt, dass die entworfenen Objekte mechanisch zuverlässig sind und nicht bei Bewegung kaputt gehen. Es generiert alle Teile, die notwendig sind um das Objekt herzustellen. Wir haben ein Schanier-System entwickelt, welches es diesen stabilen Objekten ermöglicht, sich zu bewegen. Die daraus resultierenden Teile können mit handelsüblichen 3D-Druckern erzeugt ewrden.\\
\\
Wir demonstrieren TrussFormer mit verschiedenen Beispielobjekten, u.a. einem laufenden 6-beinigen Roboter und einem 4m großen, animatronischen T-Rex.
\end{otherlanguage}

\endgroup

\vfill
