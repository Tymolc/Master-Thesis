\chapter{Introduction}\label{ch:introduction}
Personal fabrication devices, such as 3D printers, are already widely used for rapid prototyping and allow non-expert users to create interactive machines, tools and art. As consumer-grade 3D printers are usually desktop-sized, the size of these objects is, however, fairly limited. \trussFabName{} aims to enable users to create large-scale dynamic objects using desktop-sized 3D printers. Scale can be achieved by creating multiple small-sized objects and connecting them to each other. If all parts of a large object would be 3D printed, this process would take a long time and special large-size 3D printers would be needed. Our solution to this problem is to take ready-made objects, like empty plastic bottles, and only print the connectors that keep them together.\\
To aid users in this process, we developed a software simulation that can create objects which are capable of handling the substantial forces large object intrinsically have. We achieve this by providing stable primitives which can be attached together. These primitives resemble truss structures - beam-based constructions creating closed triangle surfaces, which are intrinsically sturdy and material-efficient.\\
In order to build the simulated objects, we provide export-functionalities.
Our software also provides tools to evaluate the magnitude of force acting on the links.\\
- TODO:\\
- node-link-structure\\
- export\\
- force\\
\section{TrussFab}
- create big structures\\
- create them quickly and cheaply\\
\section{TrussFormer}
- make structures move\\
- observe forces during movement\\
- create animation\\
\section{TrussControl}
- closed-loop movement control\\
- automatic conversion of simulation animation to arduino code\\
