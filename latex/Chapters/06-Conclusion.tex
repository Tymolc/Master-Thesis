\chapter{Conclusion}\label{ch:conclusion}
We created a system that enables users to design and fabricate large-scale dynamic truss structures. Our system consists of an editor, which supports creating and simulating stable truss objects with the help of a physics engine. Users can simulate motion by placing linear actuators in the structure and observe occurring forces by interactively moving the structure. Motion patterns can be created by using the keyframe animation feature. This way, the complicated motion distribution in the trusses can be thoroughly tested before building the object.\\
Our export feature enables users to generate STL files, based on the designed object, which can be 3D printed and assembled. The animation is stored in an Arduino-readable format, so the designed object can be completely recreated in the real world.\\
For assembly, users connect PET bottles to the 3D printed hubs. An ID system clarifies which of these hubs belong together.

\section{Closed-loop Positional Control}
We already made first advances towards integrating PID control into our system. In further iterations of TrussFormer, we want to increase the focus on closed-loop control mechanisms and enable users to design accurate positional actuator control. Our current approach already delivers functional PID control, however the design process is still very manual. Especially finding the correct gain values for each control term can be very challenging to users with little knowledge of PID control. We imagine an automated search process, which heuristically determines reasonable values.\\
Each control term produces a specific pattern in the motion of the controlled object. If the gain value for the proportional part is too large, the motion will oscillate around the desired position, if it is too small, the approach will be slow. An overly high integral gain value will result in the accumulated error rising very quickly, causing the controller to exert a lot of force if the desired position is far away from the actual position. This can result in an even bigger overshoot and stronger, but fewer oscillations. On the other hand, if the integral is too small, it will provide little extra control to account for residual error in the proportional part. Similarly, the derivative component will fail to control the output, while a very high gain will make it very susceptible to noisy input.\\
Based on these patterns, a procedure can be implemented that automatically fine-tunes these values.
